\documentclass{article}
\usepackage[utf8]{inputenc}
\usepackage{graphicx}
\usepackage{ragged2e}
\usepackage{array}
\usepackage{wrapfig}
\usepackage{multirow}
\usepackage{tabularx}
\usepackage{amsmath}
\usepackage{wrapfig}
\usepackage{enumitem }
\usepackage{mathtools}
\usepackage[table]{xcolor}
\usepackage{multirow}
\usepackage{polski}
\usepackage{enumitem}
\usepackage{rotating}
\usepackage{natbib}
\usepackage{graphicx}

\title{Sprawozdanie 3}
\author{Jan Bronicki 249011\\Poniedziałek, 15:15 - 16:55 TN}
\date{}


\begin{document}

\maketitle

\section{Cel ćwiczenia}

Celem ćwiczenia jest zapoznanie z językiem HTML i JavaScript, zasadami tworzenia stron
WWW i umieszczania na nich skryptów w języku JavaScript.


\section{Wstęp teoretyczny}


\subsection{}
Dokument HTML jest zwykłym plikiem tekstowym (z rozszerzeniem *.htm lub *.html), w
którym znajdują się znaczniki HTML. Znacznik (tag) HTML jest specjalnym ciągiem znaków
objętym nawiasami ostrymi. Wyróżnia się dwa typy tagów – otwierające, np. \textless h1\textgreater i zamykające,
np. \textless/H1\textgreater. Wszystkie znaczniki powinny się znaleźć w pewnych standardowych ramach. Cały
dokument powinien być objęty parą znaczników \textless HTML\textgreater \textless/HTML\textgreater. Między nimi powinna zaś się
znaleźć para znaczników \textless HEAD\textgreater\textless/HEAD\textgreater, która stanowi ramy dla informacji nagłówkowych.
Pozostałe informacje powinny być objęte z kolei znacznikami \textless BODY\textgreater\textless/BODY\textgreater. 

\subsection{Javascript}
Dzięki językowi JavaScript można wyposażyć stronę WWW w elementy dynamizmu. Skrypty
można osadzić na stronie WWW na kilka sposobów. Pierwszy z nich pozwala na osadzenie
skryptu bezpośrednio w dokumencie HTML. W poniższym przykładzie w sekcji HEAD
zdefiniowane są 2 funkcje, które są następnie wywoływane w sekcji BODY.


\section{Moja strona}
Na swojej stronie zamieściłem kilka zdjęć, linków, list, tablice, paragraf oraz dwa przyciski, których funkcjonalności zaprogramowałem ww JavaScript'cie, aby przy kliknięciu pokazywały bądź chowały zdjęciem z psem.






\end{document}



















