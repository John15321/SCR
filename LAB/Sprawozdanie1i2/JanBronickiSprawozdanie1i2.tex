\documentclass{article}
\usepackage{subcaption}
\usepackage{graphicx}
\usepackage{array}
\usepackage{wrapfig}
\usepackage{multirow}
\usepackage{tabularx}
\usepackage{amsmath}
\usepackage{wrapfig}
\usepackage{mathtools}
\usepackage{gensymb}
\usepackage[table]{xcolor}
\usepackage{xcolor,colortbl}
\usepackage{multirow}
\usepackage{polski}


\title{
    Sprawozdanie 1 i 2\\

    \Large SCR-Sieci Komputerowe,\\

    Poniedziałek 15:15-16:65
}


\author{Jan Bronicki 249011 }
\date{}

\begin{document}

\maketitle


\section{Wstęp}\par
Ćwiczenie pierwsze ma na celu zapoznanie sie z wykorzystaniem wirtualizacji systemu operacyjnego np. Ubuntu, które jest bazowane na Debianie. Wykorzystanie podstawowych poleceń Bash'a w celu administracji systemem.\par
Ćwiczenie drugie ma na celu zapoznanie się z podstawowym zarządzaniem użytkownikami na systemie Linux. Operacje takie jak tworzenie konta, dodawanie do grupy, nadawanie uprawnień itp..

\section{Wirtualizacja}
Wirtualizacja daje nam możliwość emulacji hardware'u jakiegoś komputera. Pozwala to na uruchomienie wirtualnego systemu operacyjnego wewnątrz
działającego systemu hostującego oraz programu do obsługi maszyn wirtualnych. Rozwiązanie to pozwala na testowanie działania programów na różnych
konfiguracjach sprzętowych różniących się ilością pamięci RAM oraz szybkością taktowania procesora oraz zapewniając większe bezpieczeństwo jako, że nie pracujemy zdalnie na naszym sprzęcie. \par 

Maszyna wirtualna ma także swój własny dysk, który żyje na maszynie rzeczywistej. Zwykle maszyna wirtualna nie ma dostępu do fizycznego dysku komputera, przez co można bezpiecznie testować na niej różne programy.\par

Dzięki wirtualizacji możliwe jest także jednoczesne uruchomienie kilku wirtualnych
komputerów (a co za tym idzie różnych konfiguracji oraz różnych systemów operacyjnych), połączenie ich w sieć i przesyłanie między nimi plików, zaznajamiając
się w ten sposób z protokołami sieciowymi, bez konieczności posiadania kilku maszyn
w domu. Dla potrzeb tego kursu, systemem hostującym jest Windows 10, programem
obsługującym wirtualizację VirtualBox, a systemem emulowanym - Ubuntu 18.04LTS.

\section{Dystrybucja Linuxa "Ubuntu"}
Ubuntu – kompletna dystrybucja systemu operacyjnego GNU/Linux, przeznaczona głównie do zastosowań biurowych i domowych. W 2011 dodano nową powłokę graficzną – Unity – będącą nakładką na popularnego GNOME.

Ubuntu opiera się na dystrybucji Debian (w wersji Sid) i jest rozwijane na serwisie Launchpad. Projekt sponsorowany jest przez przedsiębiorstwo Marka Shuttlewortha Canonical Ltd. oraz Ubuntu Foundation. Według sondaży online w 2012 Ubuntu był najpopularniejszą dystrybucją GNU/Linuksa na komputerach osobistych. Jednak Ubuntu jest również popularne na serwerach i w chmurze. Ubuntu wydawane jest w regularnych, sześciomiesięcznych odstępach czasowych.

\section{Konta użytkowników systemu}

\end{document}
